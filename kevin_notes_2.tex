\documentclass{article}

\usepackage{amsfonts}
\usepackage{amsmath, amssymb, amsthm}
\usepackage[margin=1in]{geometry}
\usepackage[small]{caption}
\usepackage{setspace}
\usepackage{hyperref}

\usepackage{biblatex}
\addbibresource{bibliography.bib}

\usepackage[shortlabels]{enumitem}

\usepackage[nottoc,notlot,notlof]{tocbibind}


\newtheorem{example}{Example}[section]
\newtheorem{theorem}{Theorem}[section]
\newtheorem{assumption}{Assumption}[section]
\newtheorem{lemma}{Lemma}[section]
\newtheorem{claim}{Claim}[section]
\newtheorem{definition}{Definition}[section]
\newtheorem{proposition}{Proposition}[section]
\newtheorem{corollary}{Corollary}[section]

\onehalfspacing

\DeclareMathOperator{\E}{E}
\DeclareMathOperator{\Var}{Var}
\DeclareMathOperator{\Cov}{Cov}
\DeclareMathOperator{\tr}{tr}
\DeclareMathOperator{\rank}{Rank}

\newcommand{\ta}{\tau^{t}}
\newcommand{\hta}{\hat \tau^{t}}
\newcommand{\tb}{\tau_{b}^{t}}
\newcommand{\htb}{\hat \tau_{b}^{t}}
\newcommand{\he}{\hat{\epsilon}_{i}}
\newcommand{\Yi}{Y_{i}^{obs}}
\newcommand{\Yb}{Y_{i}^{*}}
\newcommand{\hYi}{\hat{Y_{i}(0)}}
\newcommand{\us}{u_{i}^{*}}

\begin{document}

\title{Kevin's Notes on Matching Estimators for Nik}
\author{Kevin D. Duncan}
\maketitle

\begin{table}[h!]
\title{Proposed Procedure}
\begin{enumerate}
\item Estimate $\hta$
\item Get residuals $\he = \Yi-\hYi-\hta$ for each treated observation $i$. (you constantly forget the hat on the epsilons, its important to denote that its a sample residual and not from the population!)
\item For each $i$, draw $\us = \bigg\{ \begin{array}{cc} 1 & \text{ with probability } 1/2 \\ -1 & \text{with probability} 1/2 \end{array}$. I.e. the Rademacher distribution
\item Create $N_{1}$ bootstrap outcomes, $\Yb = \hYi +\hta + \us\he$
\item Estimate $\htb$ on the bootstrap sample $Z = \{\Yb, W, X\}$
\item Repeat the steps 3-5 for $b = 1,...,B$ bootstraps, and estimate $\hat \tau$ as the sample variance of $\hat \tau_{b}$ over the $b$ bootstraps.
\end{enumerate}
\label{boot}
\end{table}

\section{A Brief Rederivation of the ATET}

Some Assumptions from \cite{OnR:16} and \cite{AnI:08}

\begin{assumption}\label{ATETasmp}

\begin{enumerate}
\item Conditional on $W_{i} = w$ the sample consistss of independent draws from $Y,X \mid W = w$ for $w \in \{0,1\}$. For some $r \leq 1,\ N_{1}^{r}/N_{0} \to \theta \in (0,\infty)$
\item $X$ is continuously distributed on compact and convex support $X \subset R^{k}$. The density of $X$ is bounded and bounded away from zero on $\mathbb{X}$.
\item $W$ is independent of $Y_{i}(0)$ conditional on $X = x$ for almost every $x$. There exists a positive constant $c$ such that $P(W = 1 \mid X= x) \leq 1-c$ for almost every $x$
\item For $w = 0,1$, $\mu(w,x)$ and $\sigma^{2}(w,x)$ are Lipschitz in $\mathbb{X}$, $\sigma^{2}(w,x)$ is bounded away from zero on $\mathbb{X}$, and $\E[Y^{4} \mid W=w, X=x]$ is bounded uniformly on $\mathbb{X}$.
\end{enumerate}

\end{assumption}

For each individual we observe an outcome variable $Y_{i}^{obs}$, a set of covariates, $X_{i}$, and whether or not an individual was treated, $W_{i}$, which is either $1$, or $0$. For both the treated and untreated groups, we can define the conditional means, $\mu(0,x) = \E[Y_{i}(0) \mid X_{i} = x]\quad \mu(1,x) = \E[Y_{i}(1) \mid X_{i} = x]$. Then, for each individual $i$, the Average Treatment Effect on the Treated (ATET) is

\begin{align*} \tau_{i}^{t} &= \mu(1,x) - \mu(0,x) \end{align*}

For each group, we have $Y_{i}^{obs} = \mu(W_{i},X_{i})+e_{i}$, whether the error structure may dependend on whether or not an individual was treated or not. We can also define $\hat Y_{i}(0)$ to be the "local average" of the M closest individuals who were in the control group to the treated individual (see \cite{InR:15} or \cite{AnI:06} for a more explicit deviation of this). Assuming both groups have mean zero errors, we have the sample individual ATET,

\begin{align*} \hat \tau_{i}^{t} &= Y_{i}^{obs}-\hat Y_{i}(0) = Y_{i}(1)-\hat Y_{i}(0) \end{align*}

For finite samples the matching may not be perfect even under Assumptions (\ref{ATETasmp}. When the matching is perfect both units of this pair have covariate values equal to that of the matched unit, e.g. $X_{i} = X_{m_{i}^{c}}$. Assumptions \ref{ATETasmp} imply that as $n \to \infty$ with scalar covariates, the matching becomes precise. In finite samples, or with discrete covriates, with inexact matching, this isn't the case and $X_{i} \neq X_{m_{i}^{c}}$. Then,

$$\E(Y_{i}^{obs}-Y_{m_{i}^{c}} \mid W_{i} = 1, X_{i} = X_{m_{i}^{c}} = x) = \E(Y_{i}^{obs}-Y_{m_{i}^{c}} \mid X_{i} = X_{m_{i}^{c}} = x) = \mu(1,x) - \mu(0,x) = \tau(x)$$

Now, let $M$ be a bandwidth such that for every $j \in \{1,...,N_{c}\}$ such that $j$ is "closer" to $i$ than $M$, then, $j \in J_{M}(i)$, being the set of matched pairs. Then, when we have $X_{i} \neq X_{m_{i}^{c}}$

\begin{align*} \E_{sp}(Y_{i}^{obs}-Y_{m_{i}^{c}} \mid W_{i} = 1, X_{i}, X_{m_{i}^{c}}) &= \E_{sp}(Y_{i}^{obs}-Y_{m_{i}^{c}} \mid X_{i}, X_{m_{i}^{c}}) = \mu(1,X_{i}) - \mu(0,X_{m_{i}^{c}}) \\
&\quad= \tau(x)+ \mu(0,X_{i}) -\mu(0,X_{m_{i}^{c}}) \end{align*}

Thus we see in finite samples, when our matching isn't perfect, that there will be a residual bias term that differs from the ATET. \cite{AnI:06} show that the bias term is stochastically bounded, in particular for the ATET it is $O_{p}(N^{-r/k})$, such that if $k > 2r$ the bias term maw disrupt the convergence in distribution. Regardless, for fixed $n$, for finite samples by summing across treated individuals,

\begin{align*} \hat \tau &= N_{1}^{-1}\sum_{W_{i} = 1}\hat \tau_{i}^{t} =\tau+N_{1}^{-1}\sum_{W_{i} = 1}\mu(0,X_{i}) -\mu(0,X_{m_{i}^{c}}) \\
&\quad= \tau+N_{1}^{-1}\sum_{i}^{N}W_{i}\left(M^{-1}\sum_{j \in J_{M}(i)}(\mu(0,X_{i}) - \mu(0,X_{j}))\right)
\end{align*}

Now define, 

$$B_{n}^{t} = N_{1}^{-1}\sum_{i}^{N}W_{i}\left(M^{-1}\sum_{j \in J_{M}(i)}(\mu(0,X_{i}) - \mu(0,X_{j}))\right)$$

Then, we have no reason in finite samples that assume that 
$$B_{n}^{t}=0$$
Both \cite{OnR:16} and \cite{AnI:06} calculate $\tilde{\tau}^{t} = \hat \tau^{t}-B_{n}^{t}$. \cite{AnI:06} show that this bias term is $O_{p}(N_{1}^{-r/k})$. As a result, $\sqrt{N_{1}}B_{n}^{T}$ is not $O_{p}(1)$ in general, and if $k$ is large enough, the asymptotic distribution of $\sqrt{N_{1}}(\hta-\ta)$ is dominated by the bias term and the matching estimator in general is not $\sqrt{N_{1}}$-consistent. In the case where $k = r = 1$, then we get that $\sqrt{N_{1}}(\hta -\ta)$ will be asymptotically normal. But under Assumptions \ref{ATETasmp} we've tried to generalize this requirement, and thus (generally) require a bias corrected estimator for $\hta$. 

Under assumptions \ref{ATETasmp} we know there are consistent nonparametric estimators $\hat \mu(1,x) \to^{p} \mu(1,x)$ and $\hat \mu(0,x) \to^{p} \mu(0,x))$ for all $x$. Then, the natural estimator that arises is,

\begin{align*}\tilde \tau^{t} &= \hat\tau^{t}-\hat B_{n}^{t} \\
&= N_{1}^{-1}\sum_{W_{i} = 1}\left((Y_{i}^{obs} -M^{-1}\sum_{j \in J_{M}(i)}Y_{j}^{obs})-\left(M^{-1}\sum_{j \in J_{M}(i)}(\hat \mu(0,X_{i}) - \hat \mu(0,X_{j}))\right)\right) \\
&= N_{1}^{-1}\sum_{W_{i} = 1}\left((Y_{i}^{obs}-\hat \mu(0,X_{i}) -M^{-1}\sum_{j \in J_{M}(i)}Y_{j}^{obs}-\hat \mu(0,X_{j}))\right) \\
&= N_{1}^{-1}\sum_{W_{i} = 1}\left((Y_{i}^{obs}-\hat \mu(1,X_{i}) -M^{-1}\sum_{j \in J_{M}(i)}(Y_{j}^{obs}-\hat \mu(0,X_{j}))+\hat \mu(1,X_{i})-\hat \mu(0,X_{i})\right)
\end{align*}

Moreover, for later notation, we can define $K_{m}(i) = \sum_{l = 1}^{n}I\{i \in J_{m}(l)\}$ to be the number of times a particular observation appears in the match. $e_{i} = Y_{i}-\mu(W_{i}.X_{i})$, and $\epsilon_{i} =  \mu(1,X_{i})-\hat \mu(0,X_{i})$. Then,

\begin{align*} \tilde \tau^{t} &= N_{1}^{-1}\sum_{W_{i} = 1}\left((Y_{i}^{obs}-\hat \mu(1,X_{i}) -M^{-1}\sum_{j \in J_{M}(i)}(Y_{j}^{obs}-\hat \mu(0,X_{j}))+\hat \mu(1,X_{i})-\hat \mu(0,X_{i})\right) \\
&= N_{1}^{-1}\sum_{i=1}^{N}(W_{i}(\hat e_{i}+\hat \epsilon_{i})+(1-W_{i})M^{-1}K_{m}(i)\hat e_{i})
\end{align*}

This expression is the same as in \cite{OnR:16} equation (6). Then define $\hat e_{i} = Y_{i}^{obs}-\hat \mu(W_{i},X_{i})$, and $\epsilon_{i} = \hat \mu(1,X_{i})-\hat \mu(0,X_{i})$, such that we get

$$\tilde \tau^{t} = N_{1}^{-1}\sum_{W_{i} = 1}\left((\hat e_{i}-M^{-1}\sum_{j}\hat e_{j})+\epsilon_{i}\right)$$

\cite{OnR:16} then draw a an iid sequence $\{\eta_{i}^{*}\}_{i=1}^{N}$ such that $\E[\eta_{i} \mid Y, X< W] = 0,\ \E[\eta_{i}^{2} \mid Y, X< W] = 1,\ \E[\eta_{i}^{4} \mid Y, X< W] < \infty$ (e.g. \cite{Mam:92}) for each individual in the sample and then applies it to the above form to generate a wild bootstrap.\footnote{Ideally the distribution should match the moments of the underlying process, so mean zero, with second, third, and fourth moments being zero. }

\section{Variance}

Now we return to the proposed bootstrap procedure. A first test of its validity is matching the results of \cite{AnI:08} for their test case, their data generating process is outlined below

\begin{assumption}\label{asmp:ani}
\begin{enumerate}
\item The marginal distribution of $X$ is uniform on the interval $[0,1]$
\item The ratio of treated and control units is $N_{1}/N_{0} = \alpha$ for some $\alpha > 0$
\item The propensity score $e(x) = P(W_{i} = 1 \mid X_{i} = x)$ is constant
\item The distribution of $Y_{i}(1)$ is degenerate with $P(Y_{i}(1) = \tau) = 1$ and the conditional distribution of $Y_{i}(0)$ given $X_{i} = x$ is $N(0,1)$
\end{enumerate}
\end{assumption}

Under this framework we know that

\begin{align} \hat \tau^{t} &= N_{1}^{-1}\sum_{W_{i} = 1}Y_{i}^{obs}-\sum_{W_{i} = 0}K_{i}Y_{i} \\
&= \tau -N_{1}^{-1}\sum_{W_{i} = 0}K_{i}Y_{i} \end{align}

Conditioning on $X,W$ we know that $\E(Y_{i}(0) \mid W_{i} = 0, X) = 0$ and has conditional variance 1. And that $\E(Y_{i}(1) \mid W_{i} = 1, X) = \tau$. Then the estimatiion errors are,

\begin{align*} \he &= \Yi - \hYi-\hta \\
&= \Yi - \hYi - (N_{1}^{-1}\sum_{i}(W_{i}-(1-W_{i})K_{i})Y_{i} \\
&= \Yi - \hYi - N_{1}^{-1}\sum_{W_{i} = 1}\Yi+N_{1}^{-1}\sum_{W_{i}=0}K_{i})Y_{i}\end{align*}

\begin{align*}\E(\he \mid X,W) &= \E(\Yi - \hYi - N_{1}^{-1}\sum_{W_{i} =
1)}\Yi+N_{1}^{-1}\sum_{W_{i}=0}K_{i}Y_{i}) \\ &= \tau-0-\tau = 0\end{align*}

For the proposed bootstrap estimator, we have

\begin{align*} \hat \epsilon_{i} &= Y_{i}(1)^{obs}-\hat Y_{i}(0)-\hat \tau^{t} \\
&= \mu(x_{i},1)+e_{i}-\frac{1}{M}\sum_{j}\left(\mu(0,x_{j})+e_{j}\right)-\hat \tau^{t} \end{align*}

Then, $\Var(\he\mid X,Y,W) = \sum_{j}\Var(e_{j}\mid X,Y,W)+\Var(\hat \tau^{t}\mid X,Y,W)+2\sum_{j}\Cov(\hat \tau^{t},e_{j}\mid X,Y,W)$. As, under the assumptions, we know 

$$\Var(e_{i}\mid X,Y,W) = \begin{cases}0 & W_{i} = 1 \\ 1 & W_{i} = 1 \end{cases}$$
$$\Var(\hat \tau^{t} \mid X,Y,W) = N_{1}^{-2}\sum_{i}K_{i}^{2}$$

and $\Cov(\hat \tau^{t},e_{j}\mid X,Y,W) = N_{1}^{-1}K_{j}$, and finally, that $\sum_{W_{i}=1}\sum_{j \in J_{m}(i)}1 = \sum_{W_{i} = 0}K_{i}$. As a result, we have


\begin{align*} \Var^{*}[\hat \tau_{b}^{t} \mid X,W] &= N_{1}^{-2}\sum_{i}^{N_{1}}\E(\he^{2} \mid X,W) \\
&= N_{1}^{-2}\sum_{W_{i}=1}\left(\sum_{j \in J_{m}(i)}1+N_{1}^{2}\sum_{W_{i}=0}K_{i}^{2}+\frac{2}{N_{1}}\sum_{j \in J_{m}(i)}K_{j}\right) \\
&= N_{1}^{-2}\sum_{W_{i}=0}K_{i}+N_{1}^{-3}\sum_{W_{i}=0}K_{i}^{2}+2N_{1}^{-3}\sum_{W_{i}=0}K_{i}^{2} \\
&= N_{1}^{-2}\sum_{W_{i}=0}K_{i}+\frac{3}{N_{1}^{3}}\sum_{W_{i}=0}K_{i}^{2} \\
&= \frac{1}{N_{1}}+\frac{3}{N_{1}^{3}}\sum_{W_{i}=0}K_{i}^{2}\end{align*}

 We can compare this to the conditional variance of $\hta$,

$$\Var(\hta \mid Y,X,W) = \E(N_{1}^{-1}\sum_{i}K_{i}Y_{i} \mid Y,X,W) = N_{1}^{-2}\sum_{i}K_{i}^{2}$$

A result of this is we can show conditions where the proposed estimator will have both larger and smaller variance than the actual conditional variance. In particular, the proposed estimator will have smaller conditional variance if,

\begin{align*} N_{1}^{-2}\sum_{W_{i}=0}K_{i}^{2} &\geq \frac{1}{N_{1}}+3N_{1}^{-3}\sum_{W_{i}=0}K_{i}^{2} \\
\sum_{W_{i}=0}K_{i}^{2}\frac{N_{1}-3}{N_{1}^{3}} &\geq \frac{1}{N_{1}}
\end{align*}

But under this data generating process $\sum_{W_{i}=0}K_{i} = N_{1}$, as for each treated individual we only have one match, so we get that the proposed estimator will have smaller conditoinal variance than the DGP when

$$\sum_{W_{i}=0}K_{i}^{2} \geq \frac{N_{1}^{2}}{N_{1}-3}$$

And larger conditional variance otherwise. From the left hand term, we see that the estimator is sensitive to both the relative size of $N_{0}$ to $N_{1}$ as well as the number of matches. For a better example of this, let us look at the unconditional variance. From \cite{AnI:08} we know that

$$\E(K_{i}^{2} \mid W_{i}=0) = \frac{N_{1}}{N_{0}}+\frac{3}{2}\frac{N_{1}(N_{1}-1)(N_{0}+8/3)}{N_{0}(N_{0}+1)(N_{0}+2)}$$

Thus, the above inequality for the UNconditional variance becomes

\begin{align*} N_{0}\left(\frac{N_{1}}{N_{0}}+\frac{3}{2}\frac{N_{1}(N_{1}-1)(N_{0}+8/3)}{N_{0}(N_{0}+1)(N_{0}+2)}\right) &\geq \frac{N_{1}^{2}}{N_{1}-3} \\
\frac{N_{0}}{N_{1}^{2}}\left(\frac{N_{1}}{N_{0}}+\frac{3}{2}\frac{N_{1}(N_{1}-1)(N_{0}+8/3)}{N_{0}(N_{0}+1)(N_{0}+2)}\right) &\geq (N_{1}-3)^{-1} \\
\frac{1}{N_{1}}+\frac{3}{2}\frac{(N_{1}-1)(N_{0}+8/3)}{N_{1}(N_{0}+1)(N_{0}+2)} &\geq (N_{1}-3)^{-1} \\
\frac{N_{1}-3}{N_{1}}+\frac{3}{2}\frac{(N_{1}-3)(N_{1}-1)(N_{0}+8/3)}{N_{1}(N_{0}+1)(N_{0}+2)} &\geq 1
\end{align*}

for "large N" we have

\begin{align*} 1+\frac{3}{2}\frac{N_{1}N_{1}N_{0}}{N_{1}N_{0}N_{0}} &\geq 1 \\
1+3/2\alpha \geq 1
 \end{align*}

It is important to remember that under fixed $M$ asymptotics, that $\hat Y_{i}(0)$ is a random variable, and including it in the wild bootstrap estimator when trying to recover the variance of $Y_{i}^{obs}$ for is going to add additonal terms into the conditional variance. In this case, there are two sources of bias. The first, by adding in $Y_{i}(0)$ over the preffered $\hat \mu(1,x_{i})$, where $\hat \mu(\cdot)$ is a consistent estimator for $\mu(\cdot)$ (see \cite{AnI:06} for conditions for this to hold, or \cite{InR:15} for parameetric estimators of this mean). Secondly, we know from the large sample theory that $\hat \tau^{t}$ is also a random variable, and, since we are not imposing a "wild bootstrap" component onto either of these terms, the covariance between them further biases the conditional variance. The clear solutions to these problems has already been covered in \cite{OnR:16}.



\printbibliography


\end{document}